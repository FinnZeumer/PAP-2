\chapter{Einleitung}
\label{ch:Einleitung}

\section{Aufgabe und Motivation}
\label{sec:AuM}
\cite{skriptPAP21}
Im Rahmen des Praktikums \textit{Fourier-Optik} sollen die Beugungsphänomene eines
einfachen und eines doppelten Spalts experimentell untersucht werden.
Ziel ist es, die gemessenen Beugungsbilder mit den theoretischen Vorhersagen
der Fraunhofer-Beugung zu vergleichen und dabei explizit die
Zusammenhänge zwischen dem Spaltbild und seiner Fourier-Transformation
herauszuarbeiten.  Durch die Variation der Spaltbreite und das gezielte
Ausblenden einzelner Beugungsordnungen wird zudem demonstriert, wie
Modenfilter die räumliche Frequenzverteilung beeinflussen können.
Die gewonnenen Erkenntnisse bilden eine wichtige Grundlage für Anwendungen
wie optische Bildverarbeitung, Spektroskopie und die Entwicklung von
hochauflösenden Mikroskopen.

\section{Physikalische Grundlagen}
\label{sec:PhyBasics}
Die theoretische Beschreibung beruht auf der Fraunhofer-Beugung, bei der
die einfallenden Lichtstrahlen als ebenwellig angenommen werden.  Für einen
einfachen Spalt ergibt sich das komplexe Feld im Fernfeld zu

\begin{equation}
E(\alpha)=E_{0}\,e^{i\omega t}\int_{-d/2}^{d/2}
e^{-ik\,y\sin\alpha}\,\mathrm{d}y
\label{eq:integral_spalt}
\end{equation}

wobei $d$ die Spaltbreite, $k=2\pi/\lambda$ die Wellenzahl und $\alpha$
der Beobachtungswinkel ist.  Die Auswertung des Integrals liefert

\begin{equation}
E(\alpha)=E_{0}\,e^{i\omega t}\,d\,\operatorname{sinc}\!\left(
\frac{\pi d\sin\alpha}{\lambda}\right) .
\label{eq:sinc_spalt}
\end{equation}

Die Intensität ist das Quadrat des Betrags des Feldes:

\begin{equation}
I(\alpha)=I_{0}\,\operatorname{sinc}^{2}\!\left(
\frac{\pi d\sin\alpha}{\lambda}\right) .
\label{eq:intensitaet_spalt}
\end{equation}

Für den Doppelspalt mit Spaltbreite $d$ und Spaltabstand $g$ ergibt sich das
Feld als Superposition zweier einzelner Spalte:

\begin{equation}
E_{\text{Doppel}}(\alpha)=2E_{0}\,e^{i\omega t}\,
\cos\!\left(\frac{\pi g\sin\alpha}{\lambda}\right)\,
d\,\operatorname{sinc}\!\left(
\frac{\pi d\sin\alpha}{\lambda}\right) .
\label{eq:doppel_feld}
\end{equation}

Damit lautet die Intensitätsverteilung

\begin{equation}
I_{\text{Doppel}}(\alpha)=I_{0}\,
\cos^{2}\!\left(\frac{\pi g\sin\alpha}{\lambda}\right)\,
\operatorname{sinc}^{2}\!\left(
\frac{\pi d\sin\alpha}{\lambda}\right) .
\label{eq:doppel_intensitaet}
\end{equation}

Die Gleichungen~\ref{eq:sinc_spalt} bzw. \ref{eq:doppel_intensitaet}
zeigen deutlich, dass das Beugungsbild die Fourier-Transformation der jeweiligen
Spaltfunktion ist.  Durch das Einführen einer Modenblende in der Fourier-Ebene
können einzelne Nullstellen der \ref{eq:sinc_spalt} bzw. \ref{eq:doppel_intensitaet}
ausgespart und damit gezielt bestimmte Ordnungen unterdrückt werden.

\section{Versuchsaufbau}
\label{sec:Versuchsaufbau}
Der Versuchsaufbau ist im MEssprotokoll, sowie in
Abbildung~\ref{fig:aufbau} skizziert. Er besteht aus den folgenden Komponenten:

\begin{itemize}
  \item \textbf{Laser}: Single-Mode-Diodenlaser, Wellenlänge $\lambda\approx635\,$nm,
        mit einstellbarer Stromversorgung.
  \item \textbf{Variable Spalte}: ein verstellbarer Einfachspalt bzw. ein Doppelspalthalter,
        die als Objekt dienen.
  \item \textbf{Linse $L_{1}$} (Brennweite $f_{1}=80\,$mm):
        erzeugt die Fourier-Ebene im Abstand $f_{1}$ hinter dem Spalt.
  \item \textbf{Analysierspalt / Modenblende} in der Fourier-Ebene,
        über den einzelne Beugungsordnungen selektiv blockiert werden können.
  \item \textbf{Linse $L_{2}$} (Brennweite $f_{2}$, im Praktikum $f_{2}=100\,$mm):
        bildet das Beugungsbild auf den Detektor.
  \item \textbf{Kamera (ThorCam)}: Sensorgröße $4{,}968\,\text{mm}\times3{,}726\,\text{mm}$,
        Auflösung $1440\times1080$ Pixel, Pixelgröße $3{,}45\,\mu\text{m}$.
  \item \textbf{Strahlteiler} und \textbf{Umlenkspiegel} zur Führung des Strahls zum Detektor.
  \item \textbf{Graufilter} zur Intensitätskontrolle.
\end{itemize}

\begin{figure}[h!]
  \centering
  \includegraphics[width=0.85\linewidth]{Versuche/233/img/Versuchsaufbau-Skript.png}
  \caption{Schematischer Versuchsaufbau für die Fourier-Optik-Messungen.}
  \label{fig:aufbau}
\end{figure}

\section{Messmethodik und Datenanalyse}
\label{sec:Messmethodik}
Zur Quantifizierung der Beugungsbilder werden zunächst die Pixel-Positionen
der Minima und Maxima im Kamerabild ermittelt.  Anschließend wird eine
lineare Eichung zwischen Pixel- und Millimetereinheiten durchgeführt
($m = 0.0041\;\text{mm/px}$). Mit dieser Skalierung lässt sich der Abstand
 $d$ zwischen den Beugungsordnungen in physikalische Längen umrechnen,
woraus über die bekannten Formeln \eqref{eq:intensitaet_spalt} und
\eqref{eq:doppel_intensitaet} die Spaltbreite  $b$ bzw. der Spaltabstand
 $g$ bestimmt werden.


\section{Beugungsbild des Spalts als Fourier-Transformation der Spaltfunktion}
\subsection*{Einzelspalt}
Der Einzelspalt lässt sich durch eine rechteckige Spaltfunktion $f(y)$ beschreiben, die von der transversalen Koordinate $y$ abhängt (vgl.~Abb.~\ref{fig:eins-spalt-funktion}).  

\begin{figure}[h]
    \includegraphics[width=0.45\textwidth]{Versuche/233/img/Einfachspalt_Rechtecksfunktion.png}
    \caption{Spaltfunktion des Einfachspalts}
    \label{fig:eins-spalt-funktion}
\end{figure}

Die Fourier-Transformierte $F(k_y)$ lautet dann

\begin{equation}
    F(k_y)= A\,\operatorname{sinc}\!\left(\frac{k_y d}{2}\right),
    \label{eq:einzelspalt-ft}
\end{equation}
wobei die Nullstellen bei

\begin{equation}
    k_{y,n}= \frac{2\pi n}{d}
    \label{eq:einzelspalt-nullstellen}
\end{equation}
liegen.

Um die ursprüngliche Spaltfunktion zurückzugewinnen, muss das Inverse-Fourier-Integral ausgewertet werden. Da dieses Integral analytisch nicht lösbar ist, wird es numerisch über die symmetrische Beziehung $F(k_y)=F(-k_y)$ berechnet:

\begin{equation}
    f(y)=\frac{d}{\pi}\int_{0}^{\infty}
        \operatorname{sinc}\!\left(\frac{k_y d}{2}\right)
        \cos(k_y y)\, \mathrm{d}k_y .
    \label{eq:einzelspalt-inv}
\end{equation}

Durch Beschränkung des Integrationsbereichs bis zur $n$-ten Nullstelle $k_{y,n}$ erhalten wir ein modifiziertes Bild, das nur einen Teil des Spektrums berücksichtigt:

\begin{equation}
    f_{\text{mod}}(y)=\frac{d}{\pi}\int_{0}^{k_{y,n}}
        \operatorname{sinc}\!\left(\frac{k_y d}{2}\right)
        \cos(k_y y)\, \mathrm{d}k_y .
    \label{eq:einzelspalt-mod}
\end{equation}

Das Ergebnis wird anschließend quadriert, um die Intensitätsverteilung zu erhalten.

\subsection*{Doppelspalt}
Für den Doppelspalt wird die Spaltfunktion als Überlagerung zweier um den Abstand $g$ verschobener Einzelspalte modelliert (vgl.~Abb.~\ref{fig:doppelspalt-funktion}).  

\begin{figure}[h]
    \centering
    \includegraphics[width=0.45\textwidth]{Versuche/233/img/Doppelspalt_Spaltfunktion.png}
    \caption{Spaltfunktion des Doppelspalts}
    \label{fig:doppelspalt-funktion}
\end{figure}

Die Fourier-Transformierte ergibt sich zu

\begin{equation}
    F(k_y)=2\,d\,\cos\!\left(\frac{k_y g}{2}\right)
           \operatorname{sinc}\!\left(\frac{k_y d}{2}\right) .
    \label{eq:doppelspalt-ft}
\end{equation}

Nach Quadrieren und geeigneter Substitution $k_y = 2\pi \sin\alpha/\lambda$ erhält man die bekannte Intensitätsverteilung des Doppelspalts:

\begin{equation}
    I(\alpha)=4d^{2}\cos^{2}\!\bigl(\pi g \sin\alpha / \lambda\bigr)\;
               \operatorname{sinc}^{2}\!\bigl(\pi d \sin\alpha / \lambda\bigr) .
    \label{eq:doppelspalt-intensitaet}
\end{equation}

Analog zum Einzelspalt wird das modifizierte Bild durch Beschränkung des Integrals bis zur $n$-ten Nullstelle bestimmt:

\begin{equation}
    f_{\text{mod}}(y)=\frac{2d}{\pi}\int_{0}^{k_{y,n}}
        \cos\!\left(\frac{k_y g}{2}\right)
        \operatorname{sinc}\!\left(\frac{k_y d}{2}\right)
        \cos(k_y y)\,\mathrm{d}k_y .
    \label{eq:doppelspalt-mod}
\end{equation}

\subsection*{Beugungsbilder}

\section{Zusammenfassung des theoretischen Hintergrundes}
\label{sec:Zusammenfassung}
Die zentrale Aussage des Versuchs lautet, dass das im Fernfeld beobachtete
Beugungsbild exakt die Fourier-Transformation der räumlichen
Spaltfunktion ist.  Für einen rechteckigen Einzelspalt führt dies zu einer
 $\operatorname{sinc} $-Funktion, während beim Doppelspalt zusätzlich eine
 $\cos^{2} $-Modulation entsteht, die das Interferenzmuster zwischen den
beiden Spalten beschreibt. Durch das gezielte Blockieren einzelner
Fourier-Komponenten (Modenblende) lässt sich das Bild manipulieren und
so die Wirkung einzelner Frequenzanteile auf die Gesamtdarstellung
veranschaulichen - ein wichtiges Prinzip für moderne Techniken der
optischen Bildverarbeitung und spektralen Analyse.