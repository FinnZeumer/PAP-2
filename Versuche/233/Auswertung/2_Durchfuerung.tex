\chapter{Durchführung}
\label{ch:Durchfuehrung}

\section{Messverfahren}
\label{sec:Messverfahren}
Für die Aufnahme und Auswertung der Beugungsbilder wurde das folgende
Messprotokoll verwendet. Alle Schritte wurden gemäß den Anweisungen des
Praktikumsskripts durchgeführt und dokumentiert.

\subsection{Vorbereitung des Aufbaus}
\begin{enumerate}
  \item Der Diodenlaser wurde eingeschaltet und auf stabile Ausgangsleistung
        eingestellt. Der Graufilter wurde zunächst aus dem Strahlengang genommen,
        um eine ausreichende Intensität für die Justage zu gewährleisten.
  \item Der variable Einzelspalt (bzw. später der Doppelspalthalter) wurde als
        Objekt vor dem Laser positioniert. Die Spaltbreite wurde zunächst so gewählt,
        dass mindestens das Hauptmaximum und fünf Nebenmaxima im Bildsensor erfasst
        werden konnten (vgl. Abschnitt \ref{sec:PhyBasics}).
  \item Die Linse $L_{1}$ wurde so platziert, dass ihr Abstand zum Analysierspalt
        exakt $f_{1}=80\,$mm betrug (vgl. Abb.~\ref{fig:aufbau}).  Dieser Abstand
        definiert die Fourier-Ebene.
  \item Der Analysierspalt bzw. die Modenblende wurde in die Fourier-Ebene eingesetzt.
        Durch schrittweises Schließen des Spalts konnten die gewünschten
        Beugungsordnungen selektiv abgeschaltet werden.
  \item Die zweite Linse $L_{2}$ wurde so justiert, dass das Beugungsbild auf dem
        Detektor scharf abgebildet wird.  Der Strahlteiler und der Umlenkspiegel
        leiteten das Bild zur ThorCam-Kamera.
\end{enumerate}

\subsection{Aufnahme von Bildern}
\begin{enumerate}
  \item Die Kamera wurde über die Software \texttt{ThorCam} konfiguriert:
        \begin{itemize}
          \item Bildformat: \texttt{16-bit TIFF without Annotations}.
          \item Belichtungszeit so eingestellt, dass weder das Hauptmaximum noch
                die ersten Nebenmaxima gesättigt waren (vgl. \eqref{eq:intensitaet_spalt}).
          \item \texttt{Gain}=0.
        \end{itemize}
  \item Für jede Spaltbreite bzw. jeden Spaltabstand wurden mindestens drei
        Bilder aufgenommen:
        \begin{enumerate}
          \item Bild mit voller Beugungsordnung (keine Abschattung).
          \item Bild mit abgeschalteter Nullordnung (Modenblende blockiert $k_y=0$).
          \item Bild mit abgeschalteten höheren Ordnungen (z.\,B. $\pm1$, $\pm2$).
        \end{enumerate}
  \item Zusätzlich wurden für den Doppelspalt Bilder bei verschiedenen
        Spaltabständen $g$ aufgenommen, um die Interferenzterm
        $\cos^{2}\!\bigl(\pi g\sin\theta/\lambda\bigr)$ aus
        \eqref{eq:doppel_intensitaet} zu untersuchen.
\end{enumerate}

\subsection{Auswertung der Aufnahmen}
\begin{enumerate}
  \item Die aufgenommenen TIFF-Dateien wurden in das Analyseprogramm
        \texttt{Gwyddion} importiert. Dort wurde für jedes Bild ein
        Intensitätsprofil entlang einer horizontalen Linie durch das
        Hauptmaximum erzeugt (Menü \texttt{Intensity Profile}).
  \item Das Profil wurde als Textdatei exportiert, sodass die Intensität
        $I(x)$ in Abhängigkeit von der Pixelposition $x$ vorliegt.
  \item Die Pixelpositionen wurden mit der bekannten Pixelgröße
        $p=3{,}45\,\mu\text{m}$ in physikalische Koordinaten umgerechnet:
        $$
          X = p\cdot x .
        $$
  \item Zur Kalibrierung der Abszisse wurden die Positionen mehrerer
        Beugungsminima (mindestens fünf gut sichtbare Minima) gemessen.
        Aus den bekannten theoretischen Minima-Positionen
        $x_n = n\lambda f/(d)$ (Einzelspalt) bzw.
        $x_n = n\lambda f/g$ (Doppelspalt) wurde der Skalierungsfaktor
        bestimmt und die Bildweite $b$ sowie die Vergrößerung $M=f/b$ berechnet.
  \item Die gemessenen Intensitätsprofile wurden mit den theoretischen
        Funktionen \eqref{eq:intensitaet_spalt} bzw. \eqref{eq:doppel_intensitaet}
        verglichen.  Für die Anpassung wurden die Parameter $d$ (Spaltbreite)
        und $g$ (Spaltabstand) als freie Variablen in einem nichtlinearen
        Fit (Levenberg-Marquardt) verwendet.
\end{enumerate}

\subsection{Dokumentation}
Alle relevanten Daten wurden archiviert:
\begin{itemize}
  \item Rohbilder (TIFF) für Einzel- und Doppelspalt.
  \item Exportierte Intensitätsprofile (Textdateien).
  \item Tabellen mit gemessenen Pixel- und Real-Koordinaten der Minima.
  \item Fit-Ergebnisse inklusive Unsicherheiten für $d$, $g$, $M$ und $b$.
\end{itemize}
Diese Unterlagen bilden die Basis für die nachfolgenden Auswertungen und
Diskussionen im Protokoll.