\onecolumn

\chapter{Auswertung}
\label{ch:Auswertung}
% Liste der genutzer Formeln für die Fehlerrechnung
\section*{Fehlerrechnung}
Für die statistische Auswertung von $n$ Messwerten $x_i$ werden folgende Größen definiert \cite{errorSkript25}:
\begin{align}
    \bar{x} &= \frac{1}{n} \sum_{i=1}^{n} x_i \vphantom{\sqrt{\sum_i^n}^2} && \text{\textcolor{gray}{Arithmetisches Mittel}} \label{eq:arithmetisches_mittel} \\
    \sigma^2 &= \frac{1}{n-1} \sum_{i=1}^{n} (x_i - \bar{x})^2 \vphantom{\sqrt{\sum_i^n}^2} && \text{\textcolor{gray}{Variation}} \label{eq:variation} \\
    \sigma &= \sqrt{\frac{1}{n-1} \sum_{i=1}^{n} (x_i - \bar{x})^2} \vphantom{\sqrt{\sum_i^n}^2} && \text{\textcolor{gray}{Standardabweichung}} \label{eq:standardabweichung} \\
    \Delta \bar{x} &= \frac{\sigma}{\sqrt{n}} = \sqrt{\frac{1}{n(n-1)} \sum_{i=1}^n(\bar x - x_i)^2} \vphantom{\sqrt{\sum_i^n}^2} && \text{\textcolor{gray}{Fehler des Mittelwerts}} \label{eq:fehler_mittelwert} \\
    \Delta f &= \sqrt{\left(\frac{\partial f}{\partial x} \Delta x\right)^2 + \left(\frac{\partial f}{\partial y} \Delta y\right)^2} \vphantom{\sqrt{\sum_i^n}^2} && \text{\textcolor{gray}{Gauß'sches Fehlerfortpflanzungsgesetz für $f(x,y)$}} \label{eq:gauss_fehlfortpflanzung} \\
    \Delta f &= \sqrt{(\Delta x)^2 + (\Delta y)^2} \vphantom{\sqrt{\sum_i^n}^2} && \text{\textcolor{gray}{Fehler für $f = x + y$}} \label{eq:fehler_summe} \\
    \Delta f &= |a| \Delta x \vphantom{\sqrt{\sum_i^n}^2} && \text{\textcolor{gray}{Fehler für $f = ax$}} \label{eq:fehler_proportional} \\
    \frac{\Delta f}{|f|} &= \sqrt{\left(\frac{\Delta x}{x}\right)^2 + \left(\frac{\Delta y}{y}\right)^2} \vphantom{\sqrt{\sum_i^n}^2} && \text{\textcolor{gray}{relativer Fehler für $f = xy$ oder $f = x/y$}} \label{eq:relativer_fehler} \\
    \sigma &= \frac{|a_{lit} - a_{gem}|}{\sqrt{\Delta a_{lit}^2 + \Delta a_{gem}^2}} \vphantom{\sqrt{\sum_i^n}^2} && \text{\textcolor{gray}{Berechnung der signifikanten Abweichung}} \label{eq:signifikante_abweichung}
\end{align}

\twocolumn

Im Folgenden werden nun die Auswertungen zu den einzelnen Aufgaben durchgeführt. Die \hyperref[ch:Auswertung]{Formeln zur Auswertung} sind oben gelistet. Der \hyperref[ch:Python]{Python code} ist am Dokumentende zu finden.

\section{Aufgabe 1: Qualitative Beobachtungen am Einfachspalt}
\label{sec:A1}

Zunächst soll der Einfluss verschiedener Parameter qualitativ untersucht werden, um deren Auswirkung auf Objekt- und Beugungsbild zu verstehen.

\subsection*{Spaltbreite}
Wie zu erwarten wird bei Vergrößerung der Spaltbreite auch das \emph{Objektbild} breiter, da dies ja eben die Projektion des Spaltes ist und diesen somit \afz{darstellt}.

Das \emph{Beugungsbild} jedoch wird bei größerer Spaltbreite schmaler. Es erscheint gleichzeitig intensiver in seiner Helligkeit im Centrum, so als würde sämliches Licht des Beugungsbildes sich auf dem Mittelpunkt zentrieren.

\subsection*{Rotation des Spaltes}
Sowohl \emph{Objektbild}, als auch \emph{Beugungsbild} rotieren symmetrisch und zirkulär zur Rotation des Spaltes. 

\subsection*{Öffnung des Analysespaltes}
Die hier gemachten Beobachtungen sind nicht mehr trivial wie zuvor. Denn sobald die Spaltbreite des Analysespaltes klein genug ist, sind zwei Lichtlinien im \emph{Objektbild} zu beobachetn. 

\section{Aufgabe 2: Vermessen der Beugungsstruktur des Einfachspaltes}
\label{sec:A2}

\section{Aufgabe 3: Vermessen der Beugungsstruktur des Doppelspaltes}
\label{sec:A3}

\section{Aufgabe 4: Das Objektbild als Fouriersynthese des Beugungsbildes
am Beispiel des Einfachspaltes}
\label{sec:A4}

\section{Aufgabe 5: Das Objektbild als Fouriersynthese des Beugungsbildes
am Beispiel des Doppelspaltes}
\label{sec:A5}