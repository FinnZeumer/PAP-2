\documentclass[fontsize=10pt, twocolumn]{scrreprt}

% Gib die Versuschsnummer an, um den entsprechenden Versuchsprotokoll zu laden
\newcommand{\versuchsnummer}{233}

\usepackage{Versuchsinformationen_PAP2_Gruppe-20} % All Information about the experiment for the title
\usepackage{style} % Beinhaltet Packeges und Styling
\usepackage{short-commands-pap} % Coustum erstellte Commands, für einen besseren Workflow

\begin{document}
\begin{titlepage}
\vspace*{2cm} 
  
  \centering
    
  % Versuchs-Titel
  {\LARGE\bfseries Protokoll zum Versuch \\[0.2cm]
  \textit{\versuchsname{\versuchsnummer}} \\[0.5cm] % Vielleicht wieder Anführungszeichen hinzufügen, leider Problem, da ein Leerzeichen falsch hinzugefügt wird.
  {\large (Versuch {\versuchsnummer})}
  \vspace{1cm}}

  % Tabelle mit Metadaten
  {\large
  \begin{tabular}{@{}rl@{}}
    Autor:                  & Finn Zeumer (hz334)\\
    Versuchspatnerin:        & Annika Künstle\\[0.5em]
    Versuchsbegleiter:      & {\begleiter{\versuchsnummer}}\\[0.5em]
    Datum der Ausführung:   & {\durchfuehrungsdatum{\versuchsnummer}}\\
    \small{Abgabedatum:}    & \small{\abgabedatum{\versuchsnummer}}\\
  \end{tabular}
  }
  \vfill


  \begin{tikzpicture}[remember picture,overlay]
    \node[opacity=1,inner sep=0] at (current page.center)
      {\includegraphics[width=\paperwidth,height=\paperheight]{BG_Titelseite.pdf}};
  \end{tikzpicture}
  
\end{titlepage} 

\renewcommand{\contentsname}{Inhaltsverzeichnis}
\tableofcontents

% Lade die Versuchsnummer-spezifischen Informationen
\addcontentsline{toc}{chapter}{Messdaten} % damit trotzdem im Inhaltsverzeichnis
\label{Protokoll}

\setcounter{chapter}{2}
\addcontentsline{toc}{chapter}{\thechapter \ \ Messprotokoll} % Damit es im Inhaltsverzeichnis erscheint
\setcounter{table}{0}  % Setze den Tabellenzähler zurück
\label{Protokoll}

\thispagestyle{empty}

\includepdf[
  pages=-,               
  pagecommand={\thispagestyle{empty}} 
]{Versuche/\versuchsnummer/Messprotokoll.pdf}



\addcontentsline{lot}{table}{\protect\numberline{\thechapter.1} Eichung der Abszisse}
\addcontentsline{lot}{table}{\protect\numberline{\thechapter.2} Intensität der Doppelspaltmaxima}
\addcontentsline{lot}{table}{\protect\numberline{\thechapter.3} Spaltabstand des Analysespaltes bei maximal zugelassener Beugungsordnung am Einzelspalt}
\addcontentsline{lot}{table}{\protect\numberline{\thechapter.4} Spaltabstand des Analysespaltes bei maximal zugelassener Beugungsordnung am Doppelspalt}
\chapter{Durchführung}
\label{ch:Durchfuehrung}

\section{Messverfahren}
\label{sec:Messverfahren}


\onecolumn

\chapter{Auswertung}
\label{ch:Auswertung}
% Liste der genutzer Formeln für die Fehlerrechnung
\section*{Fehlerrechnung}
Für die statistische Auswertung von $n$ Messwerten $x_i$ werden folgende Größen definiert \cite{errorSkript25}:
\begin{align}
    \bar{x} &= \frac{1}{n} \sum_{i=1}^{n} x_i \vphantom{\sqrt{\sum_i^n}^2} && \text{\textcolor{gray}{Arithmetisches Mittel}} \label{eq:arithmetisches_mittel} \\
    \sigma^2 &= \frac{1}{n-1} \sum_{i=1}^{n} (x_i - \bar{x})^2 \vphantom{\sqrt{\sum_i^n}^2} && \text{\textcolor{gray}{Variation}} \label{eq:variation} \\
    \sigma &= \sqrt{\frac{1}{n-1} \sum_{i=1}^{n} (x_i - \bar{x})^2} \vphantom{\sqrt{\sum_i^n}^2} && \text{\textcolor{gray}{Standardabweichung}} \label{eq:standardabweichung} \\
    \Delta \bar{x} &= \frac{\sigma}{\sqrt{n}} = \sqrt{\frac{1}{n(n-1)} \sum_{i=1}^n(\bar x - x_i)^2} \vphantom{\sqrt{\sum_i^n}^2} && \text{\textcolor{gray}{Fehler des Mittelwerts}} \label{eq:fehler_mittelwert} \\
    \Delta f &= \sqrt{\left(\frac{\partial f}{\partial x} \Delta x\right)^2 + \left(\frac{\partial f}{\partial y} \Delta y\right)^2} \vphantom{\sqrt{\sum_i^n}^2} && \text{\textcolor{gray}{Gauß'sches Fehlerfortpflanzungsgesetz für $f(x,y)$}} \label{eq:gauss_fehlfortpflanzung} \\
    \Delta f &= \sqrt{(\Delta x)^2 + (\Delta y)^2} \vphantom{\sqrt{\sum_i^n}^2} && \text{\textcolor{gray}{Fehler für $f = x + y$}} \label{eq:fehler_summe} \\
    \Delta f &= |a| \Delta x \vphantom{\sqrt{\sum_i^n}^2} && \text{\textcolor{gray}{Fehler für $f = ax$}} \label{eq:fehler_proportional} \\
    \frac{\Delta f}{|f|} &= \sqrt{\left(\frac{\Delta x}{x}\right)^2 + \left(\frac{\Delta y}{y}\right)^2} \vphantom{\sqrt{\sum_i^n}^2} && \text{\textcolor{gray}{relativer Fehler für $f = xy$ oder $f = x/y$}} \label{eq:relativer_fehler} \\
    \sigma &= \frac{|a_{lit} - a_{gem}|}{\sqrt{\Delta a_{lit}^2 + \Delta a_{gem}^2}} \vphantom{\sqrt{\sum_i^n}^2} && \text{\textcolor{gray}{Berechnung der signifikanten Abweichung}} \label{eq:signifikante_abweichung}
\end{align}

\twocolumn

Im Folgenden werden nun die Auswertungen zu den einzelnen Aufgaben durchgeführt. Die \hyperref[ch:Auswertung]{Formeln zur Auswertung} sind oben gelistet. Der \hyperref[ch:Python]{Python code} ist am Dokumentende zu finden.

\section{Aufgabe 1: Qualitative Beobachtungen am Einfachspalt}
\label{sec:A1}

Zunächst soll der Einfluss verschiedener Parameter qualitativ untersucht werden, um deren Auswirkung auf Objekt- und Beugungsbild zu verstehen.

\subsection*{Spaltbreite}
Wie zu erwarten wird bei Vergrößerung der Spaltbreite auch das \emph{Objektbild} breiter, da dies ja eben die Projektion des Spaltes ist und diesen somit \afz{darstellt}.

Das \emph{Beugungsbild} jedoch wird bei größerer Spaltbreite schmaler. Es erscheint gleichzeitig intensiver in seiner Helligkeit im Centrum, so als würde sämliches Licht des Beugungsbildes sich auf dem Mittelpunkt zentrieren.

\subsection*{Rotation des Spaltes}
Sowohl \emph{Objektbild}, als auch \emph{Beugungsbild} rotieren symmetrisch und zirkulär zur Rotation des Spaltes. 

\subsection*{Öffnung des Analysespaltes}
Die hier gemachten Beobachtungen sind nicht mehr trivial wie zuvor. Denn sobald die Spaltbreite des Analysespaltes klein genug ist, sind zwei Lichtlinien im \emph{Objektbild} zu beobachetn. 

\section{Aufgabe 2: Vermessen der Beugungsstruktur des Einfachspaltes}
\label{sec:A2}

\section{Aufgabe 3: Vermessen der Beugungsstruktur des Doppelspaltes}
\label{sec:A3}

\section{Aufgabe 4: Das Objektbild als Fouriersynthese des Beugungsbildes
am Beispiel des Einfachspaltes}
\label{sec:A4}

\section{Aufgabe 5: Das Objektbild als Fouriersynthese des Beugungsbildes
am Beispiel des Doppelspaltes}
\label{sec:A5}
\chapter{Diskussion}
\label{ch:Diskussion}

\section{Zusammenfassung}
\label{sec:Zusammenfassung}

\section{Analyse der Messwerte}
\label{sec:Analyse}

\section{Kritik}
\label{sec:Kritik}

\chapter{Phython-Code}
\label{ch:Python}

Der gesamte Pythoncode ist auf auf meinem GitHub unter https://github.com/FinnZeumer/PAP-2 zu finden. Zudem ist hier auch der Souce-Code für dieses Projekt selbst, falls Interesse besteht diesen zu sehen.
\include{Versuche/\versuchsnummer/Auswertung/6_Anhang}

% Abbildungsverzeichnis
\listoffigures
\cleardoublepage

% Tabellenverzeichnis
\listoftables
\cleardoublepage

% Literaturverzeichnis 
\bibliographystyle{alpha}
\bibliography{Literaturverzeichnis-PAP2}

% Legal Reasons
\onecolumn
\small{Hinweis zur Nutzung des Universitäts-Logos sind unter: \href{https://www.uni-heidelberg.de/einrichtungen/rektorat/kum/corporatedesign/logo.html}{Nutzung des Universitätslogos} zu finden. Der rechtliche Hinweis sagt dabei: \\
"Das Logo der Universität Heidelberg steht Ihnen ausschließlich zur Nutzung für universitäre Zwecke zur Verfügung. Eine anderweitige Verwendung muss mit der Abteilung Kommunikation und Marketing abgestimmt werden. Schriftzüge und Siegel dürfen nicht verändert werden." \\ 
Es wurden alle Designrichtlinien ordnungsgemäß nach Vorgabe eingehalten.}

\end{document}